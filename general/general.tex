\documentclass[12pt]{article}

\usepackage[utf8]{inputenc}
\usepackage[T1]{fontenc}
\usepackage[a4paper,left=2cm,right=2cm,top=2.5cm,bottom=2cm]{geometry}
\usepackage{amsfonts}
\usepackage{amsmath}
\usepackage{amssymb}
\usepackage{amsthm}
\usepackage{babel}
\usepackage{hyperref}
\usepackage{cleveref}
\usepackage{color}
\usepackage{dsfont}
\usepackage{enumitem}
\usepackage{graphicx}
\usepackage{natbib}
\usepackage{pifont}

\theoremstyle{plain}% default
\newtheorem{thm}{Théorème}[section]
\newtheorem{lem}[thm]{Lemme}
\newtheorem{prop}[thm]{Proposition}
\newtheorem*{cor}{Corollaire}
\theoremstyle{definition}
\newtheorem{dfnt}{Définition}[section]
\newtheorem{exmp}{Exemple}[section]
\newtheorem{xca}[exmp]{Exercice}
\theoremstyle{remark}
\newtheorem*{rmq}{Remarque}
\newtheorem*{note}{Note}
\newtheorem{case}{Case}

\crefname{thm}{théorème}{théorèmes}
\crefname{lem}{lemme}{lemmes}
\crefname{prop}{proposition}{propositions}
\crefname{cor}{corollaire}{corollaires}
\crefname{dfnt}{définition}{définitions}
\crefname{exmp}{exemple}{exemples}
\crefname{xca}{exercice}{exercices}
\crefname{rmq}{remarque}{remarques}
\crefname{note}{note}{notes}
\crefname{case}{case}{cases}

\newcommand{\ncref}[1]{\cref{#1} "\nameref{#1}"}
\newcommand{\Ncref}[1]{\Cref{#1} \Nameref{#1}}

\begin{document}
Let define Markovian Continued Fraction (MCF) Algorithme. let $\Delta= \{ (1,\alpha,\beta) \in \mathbb{R}^3,0 \leq \beta \leq \alpha \leq 1 \}$. Let be $T : \Delta \to \Delta$ a piecewise continous map.\newline
$$ k_n
\begin{pmatrix} 1 \\ x_1^n \\ x_2^n \end{pmatrix}=
A_n \begin{pmatrix} 1 \\ x_1^{n+1} \\ x_2^{n+1} \end{pmatrix}
$$
Then $K_n \begin{pmatrix} 1 \\x_1^0 \\ x_2^0 \end{pmatrix} = S_n \begin{pmatrix}{c} 1 \\ x_1^n \\ x_2^n \end{pmatrix}$\newline
$S_n=A_1 A_2 ... A_n$ and $K_n=k_1 ... k_n$. We should name the coefficent of $S_n$:$$
S_n=\begin{pmatrix}
a_0^n & a_1^n & a_2^n \\
b_0^n & b_1^n & b_2^n \\
c_0^n & c_1^n & c_2^n
\end{pmatrix}
$$
So we have
$$
\left \{
\begin{array}{l}
a_0^n+x_1^n a_1^n +x_2^n a_2^n=K_n\\
b_0^n+x_1^n b_1^n +x_2^n b_2^n=K_n x_1^0
\end{array}
\right .
$$
Then $\frac{a_0^n+x_1^n a_1^n +x_2^n a_2^n}{b_0^n+x_1^n b_1^n +x_2^n b_2^n}=x_1^0$.\newline
We can now control the convergence $| \frac{b_0^n}{a_0^n}-x_1^0 |= | \frac{a_0^n b_0^n+x_1^n a_1^n b_1^n +x_2^n a_2^n b_2^n-a_0^n b_0^n}{a_0^n(a_0^n+x_1^n a_1^n +x_2^n a_2^n)} | \leq \frac{1}{|a_o^n|^2}(x_1^n |a_1^n b_0^n - a_0^n b_1^n |+x_2^n |a_2^n b_0^n - a_0^n b_2^n |) \leq 2 \frac{\|S_n^{-1} \|}{\| S_n \|^2}$
\section{Cas particulier}
Nous considérons le cas où les matrices de transition sont de la forme $$
A_k=\begin{pmatrix}
\alpha_k & 1 & 0 \\
\beta_k & 0 & 1 \\
1 & 0 & 0
\end{pmatrix}
$$
Alors son inverse est $$
A_k^{-1}=\begin{pmatrix}
0 & 0 & 1 \\
1 & 0 & - \alpha_k \\
0 & 1 & - \beta_k
\end{pmatrix}
$$
Soit $\alpha_k$ et $\beta_k$ deux suites de variables aléatoires i.i.d. suivant deux lois différentes. Considérons les deux suites définies pas récurence $$
\left \{
\begin{array}{l}
r_{n}=\alpha_n r_{n-1} + \beta_n r_{n-2} + r_{n-3}\\
s_{n}= - \beta_n s_{n-1} - \alpha_n s_{n-2} + s_{n-3}
\end{array}
\right .
$$
Toute la question est de savoir si $s$ est négligeable devant $r$.
\end{document}
