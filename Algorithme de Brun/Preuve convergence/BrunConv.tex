\documentclass[12pt]{article}

\usepackage[utf8]{inputenc}
\usepackage[T1]{fontenc}
\usepackage[a4paper,left=2cm,right=2cm,top=2.5cm,bottom=2cm]{geometry}
\usepackage{amsfonts}
\usepackage{amsmath}
\usepackage{amssymb}
\usepackage{amsthm}
\usepackage{babel}
\usepackage{hyperref}
\usepackage{cleveref}
\usepackage{color}
\usepackage{dsfont}
\usepackage{enumitem}
\usepackage{graphicx}
\usepackage{natbib}
\usepackage{pifont}

\theoremstyle{plain}% default
\newtheorem{thm}{Théorème}[section]
\newtheorem{lem}[thm]{Lemme}
\newtheorem{prop}[thm]{Proposition}
\newtheorem*{cor}{Corollaire}
\theoremstyle{definition}
\newtheorem{dfnt}{Définition}[section]
\newtheorem{exmp}{Exemple}[section]
\newtheorem{xca}[exmp]{Exercice}
\theoremstyle{remark}
\newtheorem*{rmq}{Remarque}
\newtheorem*{note}{Note}
\newtheorem{case}{Case}

\crefname{thm}{théorème}{théorèmes}
\crefname{lem}{lemme}{lemmes}
\crefname{prop}{proposition}{propositions}
\crefname{cor}{corollaire}{corollaires}
\crefname{dfnt}{definition}{definitions}
\crefname{exmp}{exemple}{exemples}
\crefname{xca}{exercice}{exercices}
\crefname{rmq}{remarque}{remarques}
\crefname{note}{note}{notes}
\crefname{case}{case}{cases}

\newcommand{\ncref}[1]{\cref{#1} "\nameref{#1}"}
\newcommand{\Ncref}[1]{\Cref{#1} \Nameref{#1}}


\begin{document}
\part*{Convergence of the Brun algorithm}
\section{Definition}

\begin{dfnt}
Let $\Delta=\{(x_1,x_2); 1 \geq x_1 \geq x_2 \geq 0\}$. The Brun algorithm is generated by the function $T: \Delta \to \Delta$ with
\[
\left \{
\begin{array}{r c }
T(x_1,x_2)=(\frac{1}{x_1}-N,\frac{x_2}{x_1}) \, si \, \frac{1}{x_1}-N \geq \frac{x_2}{x_1} & [j=1] \\
T(x_1,x_2)=(\frac{x_2}{x_1},\frac{1}{x_1}-N) \, si \, \frac{x_2}{x_1} \geq  \frac{1}{x_1}-N & [j=2] \\
N:=[\frac{1}{x_1}]
\end{array}
\right .
\]
\end{dfnt}
Itearing this function we get a sequence of numbers $(x_1^n,x_2^n)$ and two natural number sequences $(j(n))$ and $(N(n))$.
We could give a more linear version of this processus from the hyperplan $ \{1 \} \times \mathbb{R}^2  $ in itself.
\[
\tilde{T}:\begin{pmatrix} 1 \\ x_1^{(n)} \\ x_2^{(n)} \end{pmatrix} \mapsto \begin{pmatrix} 1 \\ x_1^{(n+1)} \\ x_2^{(n+1)} \end{pmatrix} = \left \{
\begin{array}{r c}
\frac{1}{x_1^n}\begin{pmatrix} 0 & 1 & 0 \\1 & -N(n) & 0 \\0 & 0& 1 \end{pmatrix} \begin{pmatrix} 1 \\ x_1^{(n)} \\ x_2^{(n)} \end{pmatrix} \, si \, j(n)=1\\
\frac{1}{x_1^n}\begin{pmatrix} 0 & 1 & 0 \\0 & 0 & 1 \\1 & -N(n) & 0 \end{pmatrix} \begin{pmatrix} 1 \\ x_1^{(n)} \\ x_2^{(n)} \end{pmatrix}\, si \, j(n)=2\\
\end{array}
\right .
\]
Let $Z_1^n=\begin{pmatrix} N(n) & 1 & 0 \\1 & 0 & 0 \\0 & 0 & 1 \end{pmatrix}$ and $Z_2^n=\begin{pmatrix} N(n) & 0 & 1 \\1 & 0 & 0 \\0 & 1 & 0 \end{pmatrix}$ be the inverse of the matrices above. We shall name also $S_n=Z_{j(1)}^1 Z_{j(2)}^2 ... Z_{j(n)}^n$, so we get:\[
\begin{pmatrix} 1 \\ x_1^{(0)} \\ x_2^{(0)} \end{pmatrix} = (\prod_{1 \leq i \leq n} x_1^{(i)})S_n \begin{pmatrix} 1 \\ x_1^{(n+1)} \\ x_2^{(n+1)} \end{pmatrix}
\]
We can see $S_n$ as three vectors making a cone converging to the line $D=t \times (1,x_1^{(0)},x_2^{(0)})$. \newline

\section{Geometrical point of view}

As $T$ is from $\Delta$ to $\Delta$ we can work with the basis $\begin{pmatrix} 1 \\ 0 \\ 0 \end{pmatrix},\begin{pmatrix} 1 \\ 1 \\ 0 \end{pmatrix}, \begin{pmatrix} 1 \\ 1 \\ 1 \end{pmatrix}$. In this basis we have: \[
Z_1^n \begin{pmatrix} 1 & 1 & 1 \\ 0 & 1 & 1 \\ 0 & 0 & 1 \end{pmatrix}=
\begin{pmatrix} N(n) & N(n)+1 & N(n)+1 \\ 1 & 1 & 1 \\ 0 & 0 & 1 \end{pmatrix}
\]
\[
Z_2^n \begin{pmatrix} 1 & 1 & 1 \\ 0 & 1 & 1 \\ 0 & 0 & 1 \end{pmatrix}=
\begin{pmatrix} N(n) & N(n) & N(n)+1 \\ 1 & 1 & 1 \\ 0 & 1 & 1 \end{pmatrix}
\]

So let $A_1^k,A_2^k,A_3^k$ be the non-normalised vector which are the vertice of a triangle where $(1,x_1^0,x_2^0)$ is. We have the recurrence relation:

\[j_k=1
\begin{array}{c c l}
A_1^{k+1} & = & N A_1^k+A_2^k\\
A_2^{k+1} & = & A_1^{k+1} + A_1^k \\
A_3^{k+1} & = & A_2^{k+1} + A_3^k
\end{array}
\]

\[j_k=2
\begin{array}{c c l}
A_1^{k+1} & = & N A_1^k+A_2^k\\
A_2^{k+1} & = & A_1^{k+1} + A_3^k \\
A_3^{k+1} & = & A_2^{k+1} + A_1^k
\end{array}
\]
And we know the order of the norm of this vectors.
\begin{lem}
\[
\|A_3^k \| \geq \|A_2^k\| \geq \|A_1^k \|
\]
\end{lem}
We call $D(U,V)=\|\frac{U}{\|U \|}- \frac{V}{\|V\|}\|$ and $d_k=\max(D(A_1^k,A_2^k),D(A_2^k,A_3^k),D(A_1^k),A_3^k)$. $d_k$ is the biggest length of the size of the triangle. It is a decreasing sequence bound by $0$. We should prove that it's limit is indeed $0$.\newline
We will use this useful lemma,
\begin{lem}
 $D(U,V+b U)=\frac{\|V\|}{\|V+b U\|}D(U,V)$
\end{lem}
\begin{lem}
If $j_k=2$ and $j_{k-1}=1$ then $d_{k+1} \leq \frac{2}{3} d_k$
\end{lem}
\[
\begin{array}{c c l}
D(A_1^{k+1},D(A_3^{k+1})) & = & D(N A_1^k + A_2^k,A_2^{k+1}+A_1^k)\\
 & = & D(N A_1^k + A_2^k,(N+1)a_1^k+A_2^k+A_3^k) \\
 & = & \frac{\|A_1^k+A_3^k\|}{\|(N+1)A_1^k+A_2^k+A_3^k\|}D(NA_1^k+A_2^k,A_1^k+A_3^k)\\
 & \leq & \frac{\|A_1^k+A_3^k\|}{\|A_1^k+A_2^k+A_3^k\|} d
\end{array}
\]
Yet we have $\frac{N \|A_1^k\|+\|A_2^k\|}{\|A_1^k+A_3^k\|} \geq \frac{\|A_2^k\|}{\|A_1^k+A_3^k\|} \geq \frac{1}{2}$ so \[
D(A_1^{k+1},D(A_3^{k+1})) \leq \frac{2}{3} d_k
\]
\[
\begin{array}{c c l}
D(A_1^{k+1},A_2^{k+1}) & = & D(A_1^{k+1},A_1^{k+1}+A_3^k) \\
& = & \frac{\|A_3^k\|}{\|A_3^k+A_1^{k+1}\|}D(A_1^{k+1},A_3^k)\\
& \leq & \frac{\|A_3^k\|}{\|A_3^k+N A_1^k+A_2^k\|}D(A_1^{k+1},A_3^k)\\
& \leq & \frac{1}{1+\frac{\|A_2^k\|}{\|A_3^k\|}} d\\
& \leq & \frac{1}{2} d_k
\end{array}
\]
\[
\begin{array}{c c l}
D(A_2^{k+1},A_3^{k+1}) & = & D(A_2^{k+1},A_2^{k+1}+A_1^k) \\
& = &\frac{\|A_1^k\|}{\|A_1^k+A_2^{k+1}\|} D(A_2^{k+1},A_2^{k+1}+A_1^k)\\
& \leq & \frac{1}{2} d_k
\end{array}
\]
\end{document}
