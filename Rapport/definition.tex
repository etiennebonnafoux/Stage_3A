Soit $\Delta=\{ 1 \geq x_1 \geq x_2 \geq 0 \}$, nous appelons algorithme de dévellopement en fraction continue en dimension3 une fonction $T:\Delta \to \Delta$ définie par \[ T(x_1,x_2)=(\frac{p_0+p_1 x_1+p_2 x_2}{q_0+q_1 x_1 +q_2 x_2},\frac{r_0+r_1 x_1+r_2 x_2}{q_0+q_1 x_1 +q_2 x_2}) \] avec $(p_i(x),r_i(x),q_i(x))_{i=1,2,3}\in \mathbb{N}^9$ des fonctions simultanémant constante sur des simplexes. Chaques simplexe où ces fonctions sont constantes seront appelés $\Delta_\sigma$ avec $\sigma \in \Sigma$ l'ensemble de ces simplexes. Nous avons donc
\[
\Delta = \cup_{\sigma \in \Sigma} \Delta_\sigma
\]
Et l'intersection de deux simplexes et soit vide soit un segment. Dans les deux cas de mesure nulle. Ainsi pour être tout à fait exact la fonctionT n'est définie que sur l'ensemble \[
A_0= \Delta \backslash \underset{\sigma}{\cup} Fr(\Delta_{\sigma})
\]
et en nottant par récurrence $A_k= T(A_{k-1}) \cap A_0 $ nous définissons une dynamique (non-réversible) sur un ensemble de mesure pleine. Dans la suite nous noterons aussi cette ensemble $\Delta$ bien que ça ne soit pas exact. \newline

Ainsi pour presque tout $x \in \Delta$ nous pouvons associer $\sigma(x)$ Le simplexe dans lequel il est.\newline

Il est donc naturel d'associer à $T$ les matrices $N(x)=N(\sigma(x))=\begin{pmatrix} q_0 & q_1 & q_2 \\ p_0 & p_1 & p_2 \\ r_0 & r_1 & r_2 \end{pmatrix}$. De plus nous demanderons à ce que $det(N)=1$ (c'est aussi dire que $M=N^{-1} \in SL_3{\mathbb{Z}}$ ) et nous notterons $M=N^{-1}=\begin{pmatrix} a_0 & a_1 & a_2 \\ b_0 & b_1 & b_2 \\ c_0 & c_1 & c_2 \end{pmatrix}$
%TODO notation de merde pour les coeff
\newline
Ainsi si $\tau: \Delta \to \mathbb{R}^3, (x_1,x_2) \mapsto (1,x_1,x_2)$ et $\overset{-}{.}: \mathbb{R^+}^3 \to \Delta,(x_0,x_1,x_2) \mapsto (\frac{x_1}{x_0},\frac{x_2}{x_0})$
alors $T=\overset{-}{.} \circ M \circ \tau$.\newline
Si $a(\sigma)$ est la fonction inverse de $T$ localement sur le simplexe $\Delta_{\sigma}$, $a:T(\Delta_\sigma) \to \Delta_\sigma$, nous pouvons la calculer explicitement: $a(x_1,x_2)=(\frac{b_0+b_1 x_1 + b_2 x_2}{a_0 + a_1 x_1 + a_2 x_2},\frac{c_0 + c_1 x_1 + c_2 x_2}{a_0 + a_1 x_1 + a_2 x_2})$.\newline

 Puis pour $x \in \Delta$, nous noterrons $\omega_n(x)=a_n a_{n-1} ... a_0$ avec $a_i=a(\sigma(T^i x))$\newline
ainsi nous avons $x=\omega_n T^n(x)$.
\newline
Nous appelerons $\omega_n$ le dévellopement limité de $x$ au rang $n$ et $T^n(x)$ le reste au rang $n$ de façon analogue au fraction continues en dimension $1$.

Prenons $x=(x_1,x_2) \in \Delta$. Si nous notons $M_n(x)=M(T^n(x))$, et $x^n_1,x^n_2$ les coordonées de $T^n(x)$ nous avons
\[ k_n
\begin{pmatrix} 1 \\ x_1^n \\ x_2^n \end{pmatrix}=
M_n \begin{pmatrix} 1 \\ x_1^{n+1} \\ x_2^{n+1} \end{pmatrix}
\]
avec $k_n=a_0+x^{n-1}_1 a_1 + x^{n-1}_2 a_2$.\newline

Puis $K_n \begin{pmatrix} 1 \\x_1^0 \\ x_2^0 \end{pmatrix} = S_n \begin{pmatrix} 1 \\ x_1^n \\ x_2^n \end{pmatrix}$\newline
$S_n=A_1 A_2 ... A_n$ et $K_n=k_1 ... k_n$.

We should name the coefficent of $S_n$:$$
S_n=\begin{pmatrix}
a_0^n & a_1^n & a_2^n \\
b_0^n & b_1^n & b_2^n \\
c_0^n & c_1^n & c_2^n
\end{pmatrix}
$$
So we have
$$
\left \{
\begin{array}{l}
a_0^n+x_1^n a_1^n +x_2^n a_2^n=K_n\\
b_0^n+x_1^n b_1^n +x_2^n b_2^n=K_n x_1^0
\end{array}
\right .
$$
Then $\frac{a_0^n+x_1^n a_1^n +x_2^n a_2^n}{b_0^n+x_1^n b_1^n +x_2^n b_2^n}=x_1^0$.\newline
We can now control the convergence $| \frac{b_0^n}{a_0^n}-x_1^0 |= | \frac{a_0^n b_0^n+x_1^n a_1^n b_1^n +x_2^n a_2^n b_2^n-a_0^n b_0^n}{a_0^n(a_0^n+x_1^n a_1^n +x_2^n a_2^n)} | \leq \frac{1}{|a_o^n|^2}(x_1^n |a_1^n b_0^n - a_0^n b_1^n |+x_2^n |a_2^n b_0^n - a_0^n b_2^n |) \leq 2 \frac{\|S_n^{-1} \|}{\| S_n \|^2}$\newline
So we have $\frac{\log{| \frac{b_0^n}{a_0^n}-x_1^0 |}}{n} \leq \frac{log{2}}{n}+\frac{\log {\|S_n^{-1}\|}}{n}-2\frac{\log{\|S_n\|}}{n}$ and taking the limit give $limsup \frac{\log{| \frac{b_0^n}{a_0^n}-x_1^0 |}}{n} \leq -\lambda_3-2\lambda_1=\lambda_2-\lambda_1 \leq 0 $ and the last inequalities is strict if and only if $\lambda_1 > \lambda_2$

\color{green}
We suppose that this dynamic is mixing and admit an invariant mesure $\pi$ which is regular according to the Lesbegue mesure in the following sence: there is a function $h$, analytic on the $\Delta_\sigma$ such that $h \pi= Leb$
\color{black}

\subsection{Markovian operator}
We now define the adjoint operator $P$ of $T$ on $L^1_{\pi}(\Delta)$ by \[
Pf(x)=\sum_{a(\sigma)} f(a x)\frac{h(a.x)}{h(x)}(Jac(T)_x)^{-1} \mathbf{1} _{T(\Delta_\sigma)}(x)
\]
With $Jac(T)_x$ the jacobian of $T$\newline
To shorten the notation we will write $p(x,a)=\frac{h(a.x)}{h(x)}(Jac(T(x)))^{-1} \mathbf{1}_{T(\Delta_\sigma)}(x)$\newline
We should also define $\phi(x,a)=(Jac(T(x)))^{-1} \mathbf{1}_{T(\Delta_\sigma)}(x)$ that we will use after. \newline
We have $P1=1$.
