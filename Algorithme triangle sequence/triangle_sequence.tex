\documentclass[12pt]{article}

\usepackage[utf8]{inputenc}
\usepackage[T1]{fontenc}
\usepackage[a4paper,left=2cm,right=2cm,top=2.5cm,bottom=2cm]{geometry}
\usepackage{amsfonts}
\usepackage{amsmath}
\usepackage{amssymb}
\usepackage{amsthm}
\usepackage{babel}
\usepackage{hyperref}
\usepackage{cleveref}
\usepackage{color}
\usepackage{dsfont}
\usepackage{enumitem}
\usepackage{graphicx}
\usepackage{natbib}
\usepackage{pifont}

\theoremstyle{plain}% default
\newtheorem{thm}{Théorème}[section]
\newtheorem{lem}[thm]{Lemme}
\newtheorem{prop}[thm]{Proposition}
\newtheorem*{cor}{Corollaire}
\theoremstyle{definition}
\newtheorem{dfnt}{Définition}[section]
\newtheorem{exmp}{Exemple}[section]
\newtheorem{xca}[exmp]{Exercice}
\theoremstyle{remark}
\newtheorem*{rmq}{Remarque}
\newtheorem*{note}{Note}
\newtheorem{case}{Case}

\crefname{thm}{théorème}{théorèmes}
\crefname{lem}{lemme}{lemmes}
\crefname{prop}{proposition}{propositions}
\crefname{cor}{corollaire}{corollaires}
\crefname{dfnt}{définition}{définitions}
\crefname{exmp}{exemple}{exemples}
\crefname{xca}{exercice}{exercices}
\crefname{rmq}{remarque}{remarques}
\crefname{note}{note}{notes}
\crefname{case}{case}{cases}

\newcommand{\ncref}[1]{\cref{#1} "\nameref{#1}"}
\newcommand{\Ncref}[1]{\Cref{#1} \Nameref{#1}}


\begin{document}
\part*{Convergence de l'algorithme de Brun}
\section{Definition}

Soit $\Delta=\{(x_1,x_2,x_2) \in (\mathbb{R}^+)^3,x_1+x_2+x_3=1\} $ L'algorithme des triangles est définie par \[
\begin{array}{c}
T:\Delta \to \Delta \\
x_i,x_j,x_k \mapsto x_i-x_j-N \times x_k,x_j,x_k
\end{array}
\]
Avec $\{i,j,k\}=\{1,2,3\}$, $x_i\geq x_j \geq x_k$ et $N=[\frac{x_i-x_j}{x_k}]$.
Nous noterons $C_1=\{ (x_1,x_2,x_2) \in \Delta, x_1\geq x_2 \geq x_3\}$ , $C_1=\{ (x_1,x_2,x_2) \in \Delta, x_2 \geq x_3 \geq x_1\}$ et $C_3=\{ (x_1,x_2,x_2) \in \Delta, x_3 \geq x_1 \geq x_2\}$ puis $S=C_1 \cup C_2 \cup C_3$
$A(b)=\begin{pmatrix}1 & b & 1 \\ 1 & 0 & 0 \\ 0 & 1 & 0 \end{pmatrix}$\newline
$S_n=A(b_1)A(b_2)...A(b_n)=\begin{pmatrix}a_1^n & a_2^n & a_3^n \\ b_1^n & b_2^n & b_3^n \\ c_1^n & c_2^n & c_3^n \end{pmatrix} = \left ( \begin{array}{c|c|c} e_1^n & e_2^n & e_3^n \end{array} \right ) $\newline
Posons maintenant $\beta_1^n=e_2^n \land e^3_n$, $\beta_2^n=e_3^n \land e^1_n$ et $\beta_3^n=e_1^n \land e^2_n$
Nous avons alors les relations de récurence suivantes:
\begin{center}
\begin{tabular}{c|c|c}
$a_1^{(n+1)}$ & $a_2{(n+1)}$ & $a_3^{(n+1)}$\\
\hline
$a_1^{(n)}+a_2^{(n)}$ & $N^{(n)} a_1^{(n)}+a_3^{(n)}$ & $a_1^{(n)}$ \\
\end{tabular}
\end{center}
Et ainsi $a_1^{(n+1)}=a_1^{(n)}+N^{(n-1)} a_1^{(n-1)}+a_1^{(n-2)}$.\newline
Ces relations valent aussi pour $b$, $c$ et $e$.
Posons maintenant $\beta_1^n=e_2^n \land e^3_n$, $\beta_2^n=e_3^n \land e^1_n$ et $\beta_3^n=e_1^n \land e^2_n$. Nous avons les récurences suivantes:
\begin{center}
\begin{tabular}{c|c|c}
$\beta_1^{(n+1)}$ & $\beta_2{(n+1)}$ & $\beta_3^{(n+1)}$\\
\hline
$\beta_2^{(n)}$ & $\beta_3^{(n)}$ & $-N^{(n)} \beta_3^{(n)}-\beta_2^{(n)}$
\end{tabular}
\end{center}
\section{Exposant de Lyapounov}

\section{Convergence}
\begin{lem}
$\forall n \in \mathbb{N}, \|\beta_3^{(n)}\| \leq \|e_1^{n}\|$
\end{lem}
\begin{proof}
La preuve se faite par récurence. Supposons que nous ayons $\|\beta_3^{(n)}\| \leq \|e_1^{n}\|$ alors \newline
\[
\begin{array}{c c l}
\| \beta_3^{(n+1)} \|& = & \|N^{(n)}\beta_3^{n} + \beta_3^{n-1}\| \\
& \leq & \|N^{(n)}e_1^{(n)}+e_1^{(n-1)} \| \\
& & \\
&  &
\end{array}
\]
\end{proof}
\section{$\lambda_1$ is strictly more than $0$}
In this paper \ref{MNS}, Messaoudi, Nogueria and Schweiger have schown the following theorem
\begin{thm}
For almost $\pi$-every $x$ in $\Delta$ the triangle algorithm is weakly convergent
\end{thm}
This implie that the first exponant of Lyapounov is strictly greater than $0$.
\begin{thm}
For the algorithm of the triangle partition we have $\lambda_1 > 0$
\end{thm}
\begin{proof}
We have $\lambda_1=\underset{n \to \infty}{lim}\frac{1}{n} log \| S_n \|$. And this is not influence by the choice of the norm. So we can choose the maximum of the (positif) coefficient of the matrice. We can easly have that \[
\| S_n \| = max(a_1^n,a_2^n)
\]
Then as almost every number is irrational and $\frac{b_1^n}{a_1^n} \to x$, $\frac{c_1^n}{a_1^n} \to y$,$\frac{b_2^n}{a_2^n} \to x$ and $\frac{c_2^n}{a_2^n} \to y$, we can conclude that $max(a^n_1,a^n_2) \to infty$
We have $1=\times x_i(a_1^n+x_n a_2^n+y_n a_3^2 )$
\end{proof}

\end{document}
