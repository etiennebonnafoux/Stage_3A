\documentclass[12pt]{article}

\usepackage[utf8]{inputenc}
\usepackage[T1]{fontenc}
\usepackage[a4paper,left=2cm,right=2cm,top=2.5cm,bottom=2cm]{geometry}
\usepackage{amsfonts}
\usepackage{amsmath}
\usepackage{amssymb}
\usepackage{amsthm}
\usepackage{babel}
\usepackage{hyperref}
\usepackage{cleveref}
\usepackage{color}
\usepackage{dsfont}
\usepackage{enumitem}
\usepackage{graphicx}
\usepackage{natbib}
\usepackage{pifont}

\theoremstyle{plain}% default
\newtheorem{thm}{Théorème}[section]
\newtheorem{lem}[thm]{Lemme}
\newtheorem{prop}[thm]{Proposition}
\newtheorem*{cor}{Corollaire}
\theoremstyle{definition}
\newtheorem{dfnt}{Définition}[section]
\newtheorem{exmp}{Exemple}[section]
\newtheorem{xca}[exmp]{Exercice}
\theoremstyle{remark}
\newtheorem*{rmq}{Remarque}
\newtheorem*{note}{Note}
\newtheorem{case}{Case}

\crefname{thm}{théorème}{théorèmes}
\crefname{lem}{lemme}{lemmes}
\crefname{prop}{proposition}{propositions}
\crefname{cor}{corollaire}{corollaires}
\crefname{dfnt}{definition}{definitions}
\crefname{exmp}{exemple}{exemples}
\crefname{xca}{exercice}{exercices}
\crefname{rmq}{remarque}{remarques}
\crefname{note}{note}{notes}
\crefname{case}{case}{cases}

\newcommand{\ncref}[1]{\cref{#1} "\nameref{#1}"}
\newcommand{\Ncref}[1]{\Cref{#1} \Nameref{#1}}

% TODO ajouter les opérateurs markoviens
\begin{document}
Supposons que l'on n'a déjà montrer la convergence faible d'un algorithme.
L'existence d'une mesure invariante et suffisament régulière est aussi primordiale.
Le système doit bien sûre être mélangeant.
Pour qu'un algorithme converge il faut que son opérateur markovien associé à l'extension au drapeau satisfasse les hypothèses du théorème specrtal de Ionescu-Tulcea et Marinesca.
De là nous pouvons en déduire la régularité de la solution de l'équation cohomologique.
Sous couvert de régularité de $f$ et de la nullité de son intégral nous avons donc égalité par tout. Il faut ensuite montrer l'existence d'un point fixe pour la transformation étendu au drapeau.
Puis il faut montrer que la valeur de $f$ en ce point fixe ne peut être nul ce qui amène une contradiction.
\section{Definition}
Let $\Delta=\{ 1 \geq x_1 \geq x_2 \geq 0 \}$ and $T:\Delta \to \Delta$ with \[ T(x_1,x_2)=(\frac{p_0+p_1 x_1+p_2 x_2}{q_0+q_1 x_1 +q_2 x_2},\frac{r_0+r_1 x_1+r_2 x_2}{q_0+q_1 x_1 +q_2 x_2}) \] with $(p_i,r_i,q_i)_{i=1,2,3}\in \mathbb{N}^9$ be piecewide constant function.It is natural to associate with$T $ the matrix $M=\begin{pmatrix} q_0 & q_1 & q_2 \\ p_0 & p_1 & p_2 \\ r_0 & r_1 & r_2 \end{pmatrix}$
If $\tau: \Delta \to \mathbb{R}**3, (x_1,x_2) \mapsto (1,x_1,x_2)$ and $\overset{-}{.} \mathbb{R^+}^3 \to \Delta,(x_0,x_1,x_2) \mapsto (\frac{x_1}{x_0},\frac{x_2}{x_0})$
Then $T=\overset{-}{.} \circ M \circ \tau$\newline.
Let $a$ be the locally inverse of $T$, $a:T(\Delta_a) \to \Delta_a$. Then we call $\omega_n=a_n a_{n-1} ... a_0$
\section{Extension of the algorithm at the flag}
We need to extend our definition to the flag. Let $\tilde{\Delta}=\Delta*\mathbb{P}^1$ be the set of point in $\Delta$ with a direction. We define a distance on this set by \[
\zeta=(x,u),\zeta'=(x',u'),d(\zeta,\zeta')=\delta(x,x')+\partial (u,u')
\]
with $\delta$ the euclidean distance and $\partial$ the absolute value between the sinus of the two angles.
One can show that it is indeed a distance on $\tilde{Delta}$\newline
We extend now the action of $T$ and $a$ in $\tilde{T}$ and $\tilde{a}$ with\[
\tilde{T}(x,u)=(Tx,T'_x u)
\]
We do the same with $a$.

\section{Creation of an invariante mesure}
We suppose that there is an invariant for $T$, ergodic mesure $\pi$ on $\Delta$ and that the system is mixing. Again we should extend this mesure to $\tilde{\pi}$ on $\tilde{\Delta}$.\newline
Let's recall the Oseledets Theorem
\begin{thm}
For all $x\in \Delta$ there are three vector $e^1_x,e^2_x,e^3_x$ such that \[
\begin{matrix}
1)(e^1_x,e^2_x,e^3_x)\text{is a continuous function in } \mathbb{R} \\
2)e^i_{Tx}=Be^i_{Tx} \\
3)\text{if }v=ae^1_x+be^2_x+ce^3_x\\
lim \frac{1}{n}log \|S_n^{-1}(x)v\|=
\left \{ \begin{matrix}
\lambda_1,c \ne 0\\
\lambda_2,c=0,b \ne 0 \\
\lambda_3,c=b=0,a\ne 0
\end{matrix}
\right .
\end{matrix}
\]
\end{thm}
Let $\zeta(x)$ be the contractil flag of the dynamic in $x$\newline
Then we can show that
\begin{thm}
Let $\mu$ be a probability on $\Delta$ with a support different of $p \cup \Delta$. Then for $\pi$ almost every $x$ $\zeta(x)$ is a atom of the sequence of mesure $S_n(x)\mu$
\end{thm}
\begin{dfnt}
$\tilde{\pi}(f)=\int_X f(\zeta_x) d \pi(x)$
\end{dfnt}
\begin{thm} % NB on a besoin que le system soit melangeant
$\tilde{\pi}$ is invariant by $\tilde{T}$, $\tilde{P}$ is the adjoint operator of $\tilde{T}$ on $\Delta$ with respect to $\tilde{\pi}$. $(\Delta,B_\Delta,\tilde{T},\tilde{\pi})$ is a mixing system.
\end{thm}

\section{Cohomologic equation}
\begin{thm}
Let $f \in L^1(\Delta)$, if there is $c>0$ such that $\forall n$ $|\sum_{k \leq n}f \circ T^k|\leq c$ almost everywhere and $\pi(f)=0$ then there is $g$ such that \[
f=g-g \circ T \ p.p.
\]
\end{thm}
We want to make this theorem more precise and we will need that the equallity above is true everywhere. To do so we should show a $g$ has a certain regularity. So we want to show this theorem
\begin{thm}
Let $f \in L^1_\pi(\Delta)$ be a real function such that \[
\begin{matrix}
1)\tilde{\pi}(f)=0 \\
2)\tilde{P}f \in H_\epsilon(\Delta)\\
3)\exists c \geq 0,\forall n \ for\ almost\ every\ \zeta \ \sum_{k \leq n}f \circ T^k(\zeta)\leq c
\end{matrix}
\]then there exists a unique $g \in H_\epsilon(\Delta)$ such that\[\begin{matrix}
1)\tilde{\pi}(g)=0 \\
2)f=g-g\circ T \ p.p. \\
3)\tilde{P}f=\tilde{P}g-g
\end{matrix}
\]
\end{thm}
\section{End of demonstration}

\begin{dfnt}
Let $\alpha (g,\zeta)=\frac{\| g'_x u\|**3}{|det g'_x|}$, it is a cocycle, $\alpha(g g', \zeta)=\alpha(g,g' \zeta) \alpha(g',\zeta)$.
\end{dfnt}

\begin{thm}
$\underset{n}{lim} \frac{1}{3n} log \alpha(\omega_n,\tilde{T}^n \zeta_x)=\lambda_2=0$
\end{thm}

\begin{dfnt}
Let $f(\zeta)=\frac{1}{3} log \alpha (a^x, \tilde{T} \zeta)$
\end{dfnt}

\begin{thm}
$\tilde{\pi}(f)=0=\lambda_2$
\end{thm}

\end{document}
