\documentclass[12pt]{article}

\usepackage[utf8]{inputenc}
\usepackage[T1]{fontenc}
\usepackage[a4paper,left=2cm,right=2cm,top=2.5cm,bottom=2cm]{geometry}
\usepackage{amsfonts}
\usepackage{amsmath}
\usepackage{amssymb}
\usepackage{amsthm}
\usepackage{babel}
\usepackage{hyperref}
\usepackage{cleveref}
\usepackage{color}
\usepackage{dsfont}
\usepackage{enumitem}
\usepackage{graphicx}
\usepackage{natbib}
\usepackage{pifont}

\theoremstyle{plain}% default
\newtheorem{thm}{Théorème}[section]
\newtheorem{lem}[thm]{Lemme}
\newtheorem{prop}[thm]{Proposition}
\newtheorem*{cor}{Corollaire}
\theoremstyle{definition}
\newtheorem{dfnt}{Définition}[section]
\newtheorem{exmp}{Exemple}[section]
\newtheorem{xca}[exmp]{Exercice}
\theoremstyle{remark}
\newtheorem*{rmq}{Remarque}
\newtheorem*{note}{Note}
\newtheorem{case}{Case}

\crefname{thm}{théorème}{théorèmes}
\crefname{lem}{lemme}{lemmes}
\crefname{prop}{proposition}{propositions}
\crefname{cor}{corollaire}{corollaires}
\crefname{dfnt}{définition}{définitions}
\crefname{exmp}{exemple}{exemples}
\crefname{xca}{exercice}{exercices}
\crefname{rmq}{remarque}{remarques}
\crefname{note}{note}{notes}
\crefname{case}{case}{cases}

\newcommand{\ncref}[1]{\cref{#1} "\nameref{#1}"}
\newcommand{\Ncref}[1]{\Cref{#1} \Nameref{#1}}

\begin{document}

\section{Definition}

\begin{dfnt}
Soit $B=\{(x_1,x_2); 1 \geq x_1 \geq x_2 \geq 0\}$, l'algorithme de Brun est généré par la fonction $T: \to B$ avec
$$
\left \{
\begin{array}{r c }
T(x_1,x_2)=(\frac{1}{x_1}-N,\frac{x_2}{x_1}) \, si \, \frac{1}{x_1}-N \geq \frac{x_2}{x_1} & [j=1] \\
T(x_1,x_2)=(\frac{x_2}{x_1},\frac{1}{x_1}-N) \, si \, \frac{x_2}{x_1} \geq  \frac{1}{x_1}-N & [j=2] \\
N:=[\frac{1}{x_1}]
\end{array}
\right .
$$
\end{dfnt}
En itérant la fonction, nous obtenons une suite de nombre $(x_1^n,x_2^n)$ ainsi que $j(n)$ et $N(n)$.
Nous pouvons alors donner une version plus vectoriel de ce processus allant de l'hyperplan $\mathbb{R}^2 \times \{1 \} $ dans lui même.
$$
\tilde{T}:\begin{pmatrix} 1 \\ x_1^{(n)} \\ x_2^{(n)} \end{pmatrix} \mapsto \begin{pmatrix} 1 \\ x_1^{(n+1)} \\ x_2^{(n+1)} \end{pmatrix} = \left \{
\begin{array}{r c}
\frac{1}{x_1^n}\begin{pmatrix} 0 & 1 & 0 \\0 & -N(n) & 0 \\0 & 0& 1 \end{pmatrix} \begin{pmatrix} 1 \\ x_1^{(n)} \\ x_2^{(n)} \end{pmatrix} \, si \, j(n)=1\\
\frac{1}{x_1^n}\begin{pmatrix} 0 & 1 & 0 \\0 & 0 & 1 \\0 & -N(n) & 0 \end{pmatrix} \begin{pmatrix} 1 \\ x_1^{(n)} \\ x_2^{(n)} \end{pmatrix}\, si \, j(n)=2\\
\end{array}
\right .
$$
Soit $A_1^n=\begin{pmatrix} N(n) & 1 & 0 \\1 & 0 & 0 \\0 & 0 & 1 \end{pmatrix}$ et $A_2^n=\begin{pmatrix} N(n) & 0 & 1 \\1 & 0 & 0 \\0 & 1 & 0 \end{pmatrix}$ l'inverse des deux matrices ci-dessus. Nous posons de plus $S_n=A_{j(1)}^1 A_{j(2)}^2 ... A_{j(n)}^n$, alors nous avons:$$
\begin{pmatrix} 1 \\ x_1^{(0)} \\ x_2^{(0)} \end{pmatrix} = (\prod_{1 \leq i \leq n} x_1^{(i)})S_n \begin{pmatrix} 1 \\ x_1^{(n+1)} \\ x_2^{(n+1)} \end{pmatrix}
$$
\end{document}
