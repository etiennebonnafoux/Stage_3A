%TODO Faire le plan et la bilbiographie
\documentclass[12pt]{article}

\usepackage[utf8]{inputenc}
\usepackage[T1]{fontenc}
\usepackage[a4paper,left=2cm,right=2cm,top=2.5cm,bottom=2cm]{geometry}
\usepackage{amsfonts}
\usepackage{amsmath}
\usepackage{amssymb}
\usepackage{amsthm}
\usepackage{babel}
\usepackage{hyperref}
\usepackage{cleveref}
\usepackage{color}
\usepackage{dsfont}
\usepackage{enumitem}
\usepackage{graphicx}
\usepackage{natbib}
\usepackage{pifont}

\theoremstyle{plain}% default
\newtheorem{thm}{Théorème}[section]
\newtheorem{lem}[thm]{Lemme}
\newtheorem{prop}[thm]{Proposition}
\newtheorem*{cor}{Corollaire}
\theoremstyle{definition}
\newtheorem{dfnt}{Définition}[section]
\newtheorem{exmp}{Exemple}[section]
\newtheorem{xca}[exmp]{Exercice}
\theoremstyle{remark}
\newtheorem*{rmq}{Remarque}
\newtheorem*{note}{Note}
\newtheorem{case}{Case}

\crefname{thm}{theorem}{theorems}
\crefname{lem}{lemme}{lemmes}
\crefname{prop}{proposition}{propositions}
\crefname{cor}{corollaire}{corollaires}
\crefname{dfnt}{definition}{definitions}
\crefname{exmp}{exemple}{exemples}
\crefname{xca}{exercice}{exercices}
\crefname{rmq}{remarque}{remarques}
\crefname{note}{note}{notes}
\crefname{case}{case}{cases}

\newcommand{\ncref}[1]{\cref{#1} "\nameref{#1}"}
\newcommand{\Ncref}[1]{\Cref{#1} \Nameref{#1}}

% TODO ajouter les opérateurs markoviens
%TODO la fonction doit être aussi log integrable
\begin{document}
\tableofcontents
\section*{Introduction}
The goal of this paper is to devellop a method to show that a algorithm is strongly convergent.
We suppose we already have schow that this algorithm is weakly convergent
First we have to be sure that there exists a smooth mesure for wich the dynamic is mixing.
Then we should get the condition of the Ionescu-Tulcea Marinesca theorem for the associated operator.
From there we could show the regularitie of the solution of a cohomologic equation
To finish if we can get a contradiction with a smooth enough fonction f, and a fixed point of the dynamic. The value should be null, and we show that it is not.

\section{Definition}% TODO a redefinir avec les notions de fibré du truc de Selmer
Let $\Delta=\{ 1 \geq x_1 \geq x_2 \geq 0 \}$ and $T:\Delta \to \Delta$ with \[ T(x_1,x_2)=(\frac{p_0+p_1 x_1+p_2 x_2}{q_0+q_1 x_1 +q_2 x_2},\frac{r_0+r_1 x_1+r_2 x_2}{q_0+q_1 x_1 +q_2 x_2}) \] with $(p_i(x),r_i(x),q_i(x))_{i=1,2,3}\in \mathbb{N}^9$ be piecewide constant, on some simplex, functions.Each simplex where the cooefficient are constant will be call $\Delta_\sigma$ with $\sigma \in \Sigma$ the set of all simplex. So we have
\[
\Delta = \cup_{\sigma \in \Sigma} \Delta_\sigma
\]
For almost every $x \in \Delta$ we can associate $\sigma(x)$ the simplexe in which it is (the point on the intersection of two simplex could create problem, but they have a measure null).
\newline It is natural to associate to $T$ the matrix $M(x)=M(a)=\begin{pmatrix} q_0 & q_1 & q_2 \\ p_0 & p_1 & p_2 \\ r_0 & r_1 & r_2 \end{pmatrix}$. Moreover we ask that $det(M)=1$ and we will note $M^{-1}=\begin{pmatrix} a_0 & a_1 & a_2 \\ b_0 & b_1 & b_2 \\ c_0 & c_1 & c_2 \end{pmatrix}$%TODO notation de merde pour les coeff
\newline
So if $\tau: \Delta \to \mathbb{R}^3, (x_1,x_2) \mapsto (1,x_1,x_2)$ and $\overset{-}{.}: \mathbb{R^+}^3 \to \Delta,(x_0,x_1,x_2) \mapsto (\frac{x_1}{x_0},\frac{x_2}{x_0})$
Then $T=\overset{-}{.} \circ M \circ \tau$.\newline
Let $a(\sigma)$ be the locally inverse of $T$, $a:T(\Delta_\sigma) \to \Delta_\sigma$. We can compute that $a(x_1,x_2)=(\frac{b_0+b_1 x_1 + b_2 x_2}{a_0 + a_1 x_1 + a_2 x_2},\frac{c_0 + c_1 x_1 + c_2 x_2}{a_0 + a_1 x_1 + a_2 x_2})$. Then we call $\omega_n=a_n a_{n-1} ... a_0$ with $a_i=a(\sigma(T^i x))$\newline
We have $x=\omega_n T^n(x)$.
\newline
\color{green}
We suppose that this dynamic is mixing and admit an invariant mesure $\pi$ which is regular according to the Lesbegue mesure in the following sence: there is a function $h$, analytic on the $\Delta_\sigma$ such that $h \pi= Leb$
\color{black}

\subsection{Markovian operator}
We now define the adjoint operator $P$ of $T$ on $L^1_{\pi}(\Delta)$ by \[
Pf(x)=\sum_{a(\sigma)} f(a x)\frac{h(a.x)}{h(x)}(Jac(T)_x)^{-1} \mathbf{1} _{T(\Delta_\sigma)}(x)
\]
With $Jac(T)_x$ the jacobian of $T$\newline
To abbreviate the notation we will write $p(x,a)=\frac{h(a.x)}{h(x)}(Jac(T(x)))^{-1} \mathbf{1}_{T(\Delta_\sigma)}(x)$\newline
We should also define $\phi(x,a)=(Jac(T(x)))^{-1} \mathbf{1}_{T(\Delta_\sigma)}(x)$ that we will use after. \newline
We have $P1=1$

\section{Extension of the algorithm at the flag}
We need to extend our definition to the flag. Let $\tilde{\Delta}=\Delta*\mathbb{P}^1$ be the set of point in $\Delta$ to which we attach a direction. We define a distance on this set by \[
\zeta=(x,u),\zeta'=(x',u'),d(\zeta,\zeta')=\delta(x,x')+\partial (u,u')
\]
with $\delta$ the euclidean distance and $\partial$ the absolute value between the sinus of the two angles.
One can show that it is indeed a distance on $\tilde{\Delta}$.\newline
We extend now the action of $T$ and $a$ in $\tilde{T}$ and $\tilde{a}$ with\[
\tilde{T}(x,u)=(Tx,T'_x u)
\]
We can compute \[
T'_x=(\frac{1}{q_0+q_1 x_1 + q_2 x_2})^2 \begin{pmatrix}
(p_1 q_0 - p_0 q_1)+(p_1 q_2-q_1 p_2)x_2 & (p_2 q_0-q_2 p_0)+(p_2 q_1 -q_2 p_1) x_1 \\
(r_1 q_0 - r_0 q_1)+(r_1 q_2-q_1 r_2)x_2 & (r_2 q_0-q_2 r_0)+(r_2 q_1 -q_2 r_1) x_1 \\
\end{pmatrix}
\]
Or \[
T'_x = (\frac{1}{q_0+q_1 x_1 + q_2 x_2})^2 \begin{pmatrix}
c_2-a_2 x_2 & -b_2+a_2 x_1 \\
-c_1+a_1 x_2 & b_1 - a_1 x_1 \\
\end{pmatrix}
\]
Then we have $det(T_x')= (q_0+q_1 x_1 + q_2 x_2)^{-3}$ \newline
We do the same with $a$.
\subsection{Markovian operator}
We also extend $P$ in $\tilde{P}$ by \[
\tilde{P}f(\zeta)=\sum_a f(a \zeta) p(x,a)
\]
With $a$ going on all allowed transformation.

\section{Creation of an invariante mesure}
We supposed that there is an invariant for $T$, ergodic mesure $\pi$ on $\Delta$ and that the system is mixing \ref{'BA'} %TODO donner des exemples de papiers où c'est donner
. Again we should extend this mesure to $\tilde{\pi}$ on $\tilde{\Delta}$.\newline
\subsection{Oseledets Theorem}
Let's recall the Oseledets Theorem
% TODO : parler d'extension naturel
\begin{thm}
There is $B \subset \Delta$ with $\pi(B)=1$ and $T^{-1}(B) \subset B$ such as for  $x\in B$ there are three vector $e^1_x,e^2_x,e^3_x$ of $\mathbb{R}^3$ such that \[
\begin{matrix}
1)(e^1_x,e^2_x,e^3_x)\text{is a continuous function in } \mathbb{R} \\
2)e^i_{Tx}=Be^i_{Tx} \\
3)\text{if }v=ae^1_x+be^2_x+ce^3_x\\
lim \frac{1}{n}log \|S_n^{-1}(x)v\|=
\left \{ \begin{matrix}
\lambda_1,c \ne 0\\
\lambda_2,c=0,b \ne 0 \\
\lambda_3,c=b=0,a\ne 0
\end{matrix}
\right .
\end{matrix}
\]
\end{thm}
\subsection{Invariant mesure}
Let $\zeta(x)$ be the contractil flag of the dynamic in $x$\newline
Then we can show that
\begin{thm}
Let $\mu$ be a probability on $\Delta$ with a support different of $p \cup \Delta$. Then for $\pi$ almost every $x$ $\zeta(x)$ is a atom of the sequence of mesure $S_n(x)\mu$
\end{thm}
\begin{proof}
Let $S_n^{-1}(x)=K_n D_n K_n'$ be the Cartan decomposition of the matrix $S_n^{-1}(x)$, we know that \[
((S_n^{-1}(x))^{t}
 S_n^{-1}(x))^{\frac{1}{2n}} =
 K_n^{-1'}
 D_n^{\frac{1}{2n}}
 K_n'
\]
converge $\pi$ almost surely to a symetric matrix with eigenvalues $e^{\lambda_1},e^{\lambda_2},e^{\lambda_3}$. Moreover $lim_n K'_n D_n^{\frac{1}{2n}} K_n'=k'^{-1}D K'$, $D$ is diagonal and the base $K' e_i$ is normal.
\end{proof}
\begin{dfnt}
$\tilde{\pi}(f)=\int_X f(\zeta_x) d \pi(x)$
\end{dfnt}

\begin{thm}
$\tilde{\pi}$ is invariant by $\tilde{T}$, $\tilde{P}$ is the adjoint operator of $\tilde{T}$ on $\Delta$ with respect to $\tilde{\pi}$. $(\Delta,B_\Delta,\tilde{T},\tilde{\pi})$ is a mixing system.
\end{thm}

\begin{proof}
If $f$ is in  $L_1(\tilde{\Delta})$, then the function $\phi(x)=f(\zeta_x)$ is in $L_1(\Delta)$ and we have $\tilde{\pi}(f) = \pi (\phi)$.
\newline
Then we have $\tilde{T} \zeta_x = \zeta_{\tilde{T}x}$ and $\pi$ is invariant by $T$, $P$ is the adjoint of $T$ on $\Delta$ with respect to $\pi$ and $(\Delta, B, T, \pi)$ is mixing.
\end{proof}

\section{Cohomologic equation}
\begin{thm}
Let $f \in L^1(\Delta)$, if there is $c>0$ such that $\forall n$ $|\sum_{k \leq n}f \circ T^k|\leq c$ almost everywhere and $\pi(f)=0$ then there is $g$ such that \[
f=g-g \circ T \ p.p.
\]
\end{thm}
We want to make this theorem more precise and we will need that the equallity above is true everywhere. To do so we should show a $g$ has a certain regularity. So we want to show this theorem
\begin{thm}
Let $f \in L^1_\pi(\Delta)$ be a real function such that \[
\begin{matrix}
1)\tilde{\pi}(f)=0 \\
2)\tilde{P}f \in H_\epsilon(\Delta)\\
3)\exists c \geq 0,\forall n \ for\ almost\ every\ \zeta \ \sum_{k \leq n}f \circ T^k(\zeta)\leq c
\end{matrix}
\]then there exists a unique $g \in H_\epsilon(\Delta)$ such that\[\begin{matrix}
1)\tilde{\pi}(g)=0 \\
2)f=g-g\circ T \ p.p. \\
3)\tilde{P}f=\tilde{P}g-g
\end{matrix}
\]
\end{thm}
To do so we should first show a contraction equation;
\subsection{Contraction of the operator $\tilde{P}$ }
\begin{dfnt}
If $\epsilon$ is a real strictly positiv number, let $[f]_\epsilon= sup \frac{|f(\zeta)-f(\zeta')|}{|d(\zeta,\zeta')|^\epsilon}$ \newline
We call $H_\epsilon(\Delta)$ the set of function $f$ with $[f]_\epsilon < \infty$
\end{dfnt}
We also define the norm $\| f \|=sup |f(\zeta)|$
\begin{thm}
There is a $n >0$ and $\epsilon \in ]0;1[$ and two constant
 $0 \leq \alpha < 1$ and $\beta \geq 0$ such that for all $f\in H_\epsilon(\Delta)$
 we have \[
[\tilde{P}^n f]_\epsilon \leq \alpha [f]_\epsilon +\beta \| f \|
 \]
 \end{thm}
 \begin{proof}
 We fix $\epsilon \in ]0;1[$, $n$ a natural number and $f\in H_\epsilon(\Delta)$. Let's pick a couple of flag $(\zeta, \zeta' )= ((x,u),(x',u'))$ in the same triangle. We have
\begin{flalign*}
\frac{\tilde{P}^n f(\zeta)-\tilde{P}^n f(\zeta')}{d(\zeta,\zeta')^\epsilon} & \leq  \sum_{\omega_n} \frac{|f(\omega_n \zeta)p(x,\omega_n)-f(\omega_n \zeta')p(x',\omega_n)|}{d(\zeta,\zeta')^\epsilon} \\
&\leq  [f]_\epsilon \sum_{\omega_n} (\frac{d(\omega_n \zeta, \omega_n \zeta')}{d(\zeta,\zeta')})^\epsilon p(x,\omega_n) + \| f \| \sum_{\omega_n} \frac{|p(x,\omega_n)-p(x',\omega_n)|}{d(\zeta,\zeta')^\epsilon}
\end{flalign*}

 We should know with the two terms. For the second one we have
\begin{flalign*}
\frac{|p(x,\omega_n)-p(x',\omega_n)}{d(\zeta,\zeta' )^\epsilon} & \leq  \frac{|p(x,\omega_n)-p(x',\omega_n)|}{\delta(x,x ')^\epsilon} \\
 & \leq  C \frac{| \phi(x,\omega_n)-\phi(x',\omega_n)|}{\delta(x,x')^\epsilon}+ \frac{\phi(x,\omega_n)}{\delta(x,x')^\epsilon}|\frac{h(\omega_n x)}{h(x)}-\frac{h \omega_n x'}{h(x')}| \\
 & \leq   C \frac{| \phi(x,\omega_n)-\phi(x',\omega_n)|}{\delta(x,x')^\epsilon} + \frac{1}{c}\phi(x,\omega_n) \delta(x,x')^{1-\epsilon}[h](1+\rho^n)
\end{flalign*}

 With $[h]$ the Lipschitz coefficient of $h$, $c$ a non negative under bound of $h$, $C$ a upper bound of $\frac{h(x)}{h(y)}$,
 \color{green}And $\rho$ is a contraction constant over all function $a$ \color{black}.\newline
 $\phi(.,\omega_n)$ are differential over $X$ and we have \[
[\phi(.,\omega_n)]=\underset{x \ne x'}{sup} \frac{|\phi(x,\omega_n)-\phi(x',\omega_n)|}{\delta(x,x')} \leq \frac{c'}{(q^1_{i_1}...q^n_{i_n})^4}
 \]
 With $q^j_{i_j}= \underset{i_j=1,2,3}{max}(q^j_{i_j})$. Therefor
 \begin{flalign*}
\sum_{\omega_n} \frac{| \phi(x,\omega_n)-\phi(x',\omega_n)|}{d(\zeta,\zeta')^\epsilon}& \leq  \underset{x \ne x'}{sup} \  \partial(x,x')^{1-\epsilon} \sum_{\omega_n} \frac{c'}{(q^1_{i_1}...q^n_{i_n})^4} \\
& \leq  C_\epsilon (\sum_{a \geq 1} \frac{a+1}{a^4})^n
\end{flalign*}

 This last sum in converging.As the sum $\sum_{\omega_n} \phi(x,\omega_n)$ is also converging we have shown the existence of $\beta \geq 0$ and this for every $\epsilon$ and $n$.\newline
 We can now analyse the first term.\begin{flalign*}
(\frac{d(\omega_n \zeta,\omega_n \zeta')}{d(\zeta,\zeta')})^\epsilon & =  (\frac{\partial(\omega_n x ,\omega_n x')+\delta((\omega_n')_x u,(\omega_n')_x' u'}{\partial(x,x')+\delta(u,u')})^\epsilon\\
& \leq \frac{\partial(\omega_n x ,\omega_n x')}{\partial(x,x')}^\epsilon+ \frac{\delta((\omega_n')_x u,(\omega_n')_x' u')}{\delta(u,u')}^\epsilon \\
& \leq  \rho^{n \epsilon}+ \frac{\delta((\omega_n')_x u,(\omega_n')_x' u')}{\delta(u,u')}^\epsilon \\
\end{flalign*}
So now we have to show that we can choose $\epsilon$ in $]0;1[$ and $n$ in $\mathbb{N}*$ such that \[
\rho^{n \epsilon} + \underset{(\zeta,\zeta')\in M}{sup} \sum_{\omega_n}
\frac{\delta((\omega_n')_x u,(\omega_n')_x' u')}{\delta(u,u')}^\epsilon p(x,\omega_n) < 1
\]
We will use the Lyapounov exponant associated with $T$. We call $M$ the set of all couple of flag $(\zeta,\zeta')=((x,u),(x',u'))$ which have their origin in the same simplex and have $u \ne u'$.\newline
If we call $\mathcal{P}_2$the set of projective map from $\mathbb{R}^2$ in $\mathbb{R}$. We define the map $\theta$ from $\mathcal{P} \times M$ in$\mathbb{R}_+$ by \[
\theta(a,(\zeta,\zeta'))=\frac{\delta(a'_x u , a'_{x'} u')}{\delta(u,u')}
\]
Then $\theta$ is a multiplicative cocycle. Indeed we have $(ab)'_x = a'_{bx}b'_x$ so \[
\theta(ab,(\zeta,\zeta'))=\theta(a,(b \zeta,b \zeta')) \theta(a,(\zeta,\zeta'))
\]
Now we compactified $M$ in $\bar{M}$ by adding $M'={(x,u)}$. $M$ is an dense open of $\bar{M}$ and we can extend $\theta$ to it.\[
\theta(a,(x,v))=\theta(a,\zeta=\frac{|det(a'x)|}{\| a'_x u \|^2}
\]
with $\|u\|=1$. Then for every $n$ there is $\zeta_n$ such that \[
\underset{(\zeta,\zeta')\in M}{sup} \sum_{\omega_n} \theta(\omega_n,(\zeta,\zeta'))^\epsilon p(x,\omega_n)= \sum_{\omega_n} \theta(\omega_n,\zeta_n)^\epsilon p(x_n,\omega_n)
\]
For every real $x$, we have $e^x \leq 1+ x + \frac{1}{2}x^2 e^{|x|}$ so for every $n$ we have: \begin{flalign*}
u_{n,\epsilon} & =  \sum_{\omega_n} \theta(\omega_n,\zeta_n)^\epsilon p(x_n,\omega_n) \\
& \leq  1+ \epsilon \sum_{\omega_n} log\theta(\omega_n,\zeta_n)p(x_n,\omega_n)+\frac{\epsilon^2}{2}
\sum_{\omega_n} (log \theta (\omega_n,\zeta))^2 e^{\epsilon |log \theta (\omega_n,\zeta)|}p(x,\omega)
\end{flalign*}
 \begin{lem}
 The sum \[
 \sum_{\omega_n} (log \theta (\omega_n,\zeta))^2 e^{\epsilon |log \theta (\omega_n,\zeta)|}p(x,\omega)
 \]
 converges uniformly on $\bar{M}$
 \end{lem}

\begin{proof}
We have \[
\theta(\omega_n,\zeta) = \frac{|det((\omega_n')_x)|}{\| (\omega_n')_x^{-1} \|}
\]
As $\omega_n$ is inversible we have \[
\| (\omega_n)'_x u \| \geq \frac{\| u \|}{\|(\omega_n')_x^{-1} \|}
\]
The we get \[
\theta(\omega_n,(\zeta,\zeta')) \leq \underset{x}{sup} |det (\omega_n')_x| \| (\omega_n')_x^{-1} \|^2
\]
And we got
\[
|det (\omega_n)'_x| = \frac{1}{(q_n+q_{n-1} x^1 +q_{n-2} x^2)^3} \leq \frac{1}{q_{i_n}}
\]
With $q_{i_n}=max(q_1,q_2,q_3)$.
Moreover we have \[
\| |det (\omega_n)'_x| \| \leq 2 \| \Lambda^2 S_n \| \|S_n \| \leq 2K \| S_n \|^2
\]%NB pas clair il faudrait revoir ça
(As $\lambda_2 \leq 0$ we can make the last inegualitie).\newline
To sum up we have $\theta(\omega_n,\zeta) \leq C q_{i_n}$ then
\begin{flalign*}
 \sum_{\omega_n} (log \theta (\omega_n,\zeta))^2 e^{\epsilon |log \theta (\omega_n,\zeta)|}p(x,\omega) & \leq  C_\epsilon \sum_{\omega_n} (log(C q_n))^2 q_n^\epsilon p(x,\omega_n) \\
 & \leq  C_\epsilon \sum_{q_n} \frac{\log(C q_n)^2}{q_n^{3-\epsilon}}
\end{flalign*}

This last sum converge if $\epsilon < 1$.
\end{proof}


\begin{lem}
For $\zeta=(x,u)$ in $\Delta$ we can define $f$ with \[
f(\zeta)=\sum_a log \theta(a,\zeta)p(x,a)
\]
Then we get \[
\sum_{\omega_n} log \theta(\omega_n,\zeta_n)p(x_n,\omega_n)=\sum_{k=0}^{n-1} \tilde{P}^k f(\zeta_n)
\]
\end{lem}
\begin{proof}
We write $\omega_n =a_1 a_2 ... a_n$. AS $\theta$ is a cocycle we have \[
log \theta(\omega_n , \zeta_n)=\sum_{k=1}^n log \theta(a_k,a_{k+1}...a_n \zeta_n)
\]
And we have for all $k$:\[
p(x_n,\omega_n)=p(a_2...a_n x_n,a_1)...p(a_{k+1}...a_n x_n,a_k)...p(x_n , (a_{k+1} ... a_n))
\]
So we get: \[
\sum_{\omega_n}  log \theta(\omega_n,\zeta_n)p(x_n,\omega_n)
= \sum_{k=1}^n
 \sum_{\omega'_{n-k}}(\sum_{a_k}
 log \theta(a_k,\omega'_{n-k} \zeta_n)
 p(\omega'_{n-k}, a_k)) p(x_n,\omega'_{n-k})
\]
With $\omega'_{n-k}=(a_{k+1}...a_n)$ and as \[
\sum_{a_1}...\sum_{a_{k-1}}p(a_2... a_n x_n ,a_1)...p(a_k...a_n x_n , a_{k-1}) =1
\]
So \[
\sum_{\omega_n}  log \theta(\omega_n,\zeta_n)p(x_n,\omega_n) = \sum_{k=1}^n \sum_{\omega'_{n-k}} f(\omega'_{n-k} \zeta_n) p(x_n, \omega'_{n-k})=\sum_{k=0}^{n-1} \tilde{P}^k f(\zeta_n)
\]
\end{proof}
\begin{lem}
There is a real $\gamma$, $- \lambda_1 \leq \gamma < 0$ such for $n$ large enough we have \[
\sum_{\omega_n}  log \theta(\omega_n,\zeta_n)p(x_n,\omega_n) \leq n \gamma < 0
\]
\end{lem}

\begin{proof}%TODO voir si on met le lemme 4.1
$f$ is continuous in $\tilde{\Delta}$ so we have that $\frac{1}{n} \sum_{k=0}^{n-1} \tilde{P}^k f(\zeta)$ converge uniformly to $\tilde{\pi}(f)$ Moreover \[
\tilde{\pi}(f)=\int_X f(\zeta_x)d \pi(x) = \int_X \sum_a log \theta (a,\zeta_x)p(x,a) d \pi(x)
\]
If we note $g$ the function $g(\zeta)=log \theta(a^x,\tilde{T}\zeta)$ if $\zeta=(x,u)$ we have $\tilde{P}g(\zeta)=f(\zeta)$ and \[
\tilde{\pi}(f)=\int_X \tilde{P}g(\zeta_x) d \pi(x) = \int_X g(\zeta_x) d \pi(x)= \lambda_3 - \lambda_2 = - \lambda_1 < 0
\]
because we have $\lambda_2=0$ and $\lambda_1 > 0$. %NB ou simplicite de spectre
 We can conclude that there is a real $\gamma$, $- \lambda_1 < \gamma < 0$ such that eventually we have \[
 \sum_{k=0}^{n-1}\tilde{P}^k f(\zeta_n) \leq n \gamma <0
 \]
\end{proof}
We can now end the demonstration of theorem 7.1. We choose $n$ big enough for the inequality of the previous lemma to be true and we choose $\epsilon$ in $]0;1[$ small enough for \begin{flalign*}
1) & -1 \leq n \gamma \epsilon < 0 \\
2) & \rho^{n \epsilon}+ \frac{1}{2} \epsilon^2 \sum_{\omega_n} [log \theta(\omega_n,(\zeta,\zeta'))]^2 e^{\epsilon |log \theta(\omega_n,(\zeta,\zeta'))|}p(x,\omega_n) << n \gamma \epsilon
\end{flalign*}
We can conclude \[
\rho^{n\epsilon} + sup \sum_{\omega_n}
\frac{\delta((\omega_n')_x u,(\omega_n')_x' u')}{\delta(u,u')}^\epsilon p(x,\omega_n) \leq \alpha < 1
\]
 \end{proof}

\subsection{Proof of the cohomologic equation}
Now we can show the theorem 4.2
\begin{proof}
As for every $f$ in $H_\epsilon(\Delta)$ we have: \[
[\tilde{P}^n f]_\epsilon \leq \alpha [f]_\epsilon +\beta \| f \|
\]
The theorem of Ionescu-Tulcea and Marinescu \cite{ITM} give the spectral decomposition of the operator $\tilde{P^n}$ in $H_\epsilon(\Delta)$: there is a finites number of eighenvalue of module $1$ who have finite multiplicity and the other part of the spectra is of module strictly less than $1$. We can also write: \[
\tilde{P^n}f=\sum_{i=1}^p \lambda_i \tilde{P_i}f + \tilde{R}f
\]
where $\lambda_i$ are the eighenvalue with module $1$, $\tilde{P_i}$ are the projection on the eighen spaces who have finite dimension and $\tilde{R}$ is a linear operator of $H_\epsilon(\Delta)$ with spectral radius $r<1$ (according to the ITM theorem).\newline
As the dynamical system $(\Delta,B_\Delta,\tilde{T,\tilde{\pi}})$ is mixing and $\tilde{P}$ is the adjoint of $\tilde{T}$ according to $\tilde{\pi}$ we can conclude that $1$ is the unique eigen value of module $1$ of $\tilde{P}$ and it is simple. So for every function $f$ in $H_\epsilon(\Delta)$, we can write:\[
\tilde{P^n}f=\tilde{\pi}(f)+\tilde{R}f
\]
with $\| \tilde{R^m}f\| \leq r^m \|f\|$\newline
So if we got two solution $g$ and $h$ of $\tilde{P}f=\tilde{P}g-g$ with $\tilde{\pi}(g)=\tilde{\pi}(h)=0$ we have: \[
\tilde{P}(g-h)=(g-h)
\]
So $g-h$ is constant and therefor equal to 0.\newline
Let $f$ be a function such that $\tilde{P}f$ is in $H_\epsilon(\Delta)$ and is integral is zero, we can find asolution of the equation:\[
\tilde{P}f=\tilde{P}g-g
\]
Indeed the solution should be $g=-\sum_{k=1}^{\infty} \tilde{P^k}f$, we have to prove that $g\in H_\epsilon(\Delta)$.\newline
As $\tilde{\pi}(f)=0$ we have $\tilde{P}^{nm} f=\tilde{R}^m f$ so\[
\|\tilde{P}^{nm} f\| \leq r^m \|f \|
\]
Let $k$ and $j$ be natural numbers we have
 \begin{flalign*}
\ [\tilde{P}^{jn+k}f]_\epsilon & \leq  \alpha [\tilde{P}^{(j-1)n+k}f]_\epsilon + \beta \| \tilde{P}^{(j-1)n+k}f \| \\
& \leq  \alpha (\alpha [\tilde{P}^{(j-2)n+k}f]_\epsilon + \beta \| \tilde{P}^{(j-2)n+k}f \|) + \beta r^{j-1}\| \tilde{P}^k f \| \\
& \leq  \alpha^j [\tilde{P}^k f]_\epsilon + \alpha^{j-1} \beta  \| \tilde{P}^k f \| + ... + \beta r^{j-1} \| \tilde{P}^k f \| \\
& =  \alpha^j [\tilde{P}^k f]_\epsilon + \beta \| \tilde{P}^k f \| \sum_{i=0}^{j-1} \alpha^{j-1-i}r^i
\end{flalign*}
So if we could choose $r < \alpha$ we have \[
[\tilde{P}^{jn+k}f]_\epsilon \leq \alpha^j ([\tilde{P}^k f]_\epsilon+\| \tilde{P}^k f \| \frac{\beta}{\alpha - r})
\]
Then for every $N$ \begin{flalign*}
\ [ \sum_{k=1}^N \tilde{P}^k f ]_\epsilon & \leq  \sum_{k=1}^N [\tilde{P}^k f]_\epsilon \\
& \leq  \sum_{k=1}^{n(1+\lfloor \frac{N}{n} \rfloor)} [\tilde{P}^k f]_\epsilon \\
& \leq  \sum_{k=0}^{n-1} \sum_{j=1}^{1+\lfloor \frac{N}{n} \rfloor} [\tilde{P}^{jn+k} f]_\epsilon\\
& \leq  \sum_{k=0}^{n-1} \sum_{j=1}^{1+\lfloor\frac{N}{n} \rfloor} \alpha^j ([\tilde{P}^k f]_\epsilon+\| \tilde{P}^k f \| \frac{\beta}{\alpha - r}) \\
& \leq  \frac{1}{\alpha} \sum_{k=0}^{n-1} [\tilde{P}^k f]_\epsilon+\| \tilde{P}^k f \| \frac{\beta}{\alpha - r}
\end{flalign*}
Which is independant of $N$. So $g$ is in $H_\epsilon(\Delta)$.
\end{proof}

\section{End of demonstration}

\begin{dfnt}
Let $\alpha (g,\zeta)=\frac{\| g'_x u\|^3}{|det g'_x|}$, it is a cocycle, $\alpha(g g', \zeta)=\alpha(g,g' \zeta) \alpha(g',\zeta)$.
\end{dfnt}

\begin{thm} For $\tilde{\pi}$ almost every $\zeta=(x,u)$ if $\omega_n=a_1 ... a_n$ are the first term of the decomposition we have \[
\underset{n}{lim} \frac{1}{3n} log \alpha(\omega_n,\tilde{T}^n \zeta_x)=\lambda_2=0
\]
\end{thm}

\begin{proof}
We have $\omega_n x = S_n x$ and $|det (\omega_n)'_x |=\frac{1}{(q_n+q_{n-1}x^1+q_{n-2}x^2)^3}$. As $\tilde{\pi}$ is concentred on the couples $(x,\zeta_x)$ with $\zeta_x=(e^1_x,e^1_x \land e^2_x)$ we have \[
\alpha(\omega_n,\tilde{T^n}\zeta_x)=\frac{\| (\omega_n)'_x  e^1_x \land e^2_x\|}{| det (\omega_n)'_x |}
\]
We can conclude with theorem 3.1 .
\end{proof}

\begin{dfnt}
Let $f(\zeta)=\frac{1}{3} log \alpha (a^x, \tilde{T} \zeta)$
\end{dfnt}

\begin{thm}
$\tilde{\pi}(f)=0=\lambda_2$
\end{thm}

\begin{proof}
By the ergodique theorem we have \[
\frac{1}{n} \sum_{k=0}^n f \circ T^k (\zeta) \to \tilde{\pi}(f)
\]
And we also have
\begin{flalign*}
\frac{1}{n} \sum_{k=0}^n f \circ T^k (\zeta) & =  \frac{1}{3n} \sum_{k=0}^{n-1} log \frac{ \| (a_{T^k x})' \tilde{T^{k+1}} \zeta \| }{| det a_x'|}\\
 &= \frac{1}{3n} log \prod_{k=0}^{n-1} \frac{ \| (a_{T^k x})' \tilde{T^{k+1}} \zeta \| }{| det a_x'|} \\
 &=  \frac{1}{3n} log \alpha(\omega_n,\tilde{T}^n \zeta_x)
\end{flalign*}

\end{proof}

To conclude we should have the following hypothetises
\begin{itemize}
\item $f$ is continuous and $f \in L^2_{\tilde{\pi}}(\Delta)$
\item $\tilde{P}f \in H_\epsilon (\Delta)$
\item There is a fixed point of the dynamic $\zeta_0$
\end{itemize}
Then according to the theorem $4.2$ there exists $g$ in $H_\epsilon(\Delta)$ such as $f=g-g \circ T$ almost everywhere. With the continuity of $f$ and $g$ this equality become true everywhere.\newline
So we have $f(\zeta_0)=0$. We just need to verify that is it not true.

\bibliographystyle{plain}
\bibliography{./biblio.bib}
\end{document}
