\documentclass[12pt]{article}

\usepackage[utf8]{inputenc}
\usepackage[T1]{fontenc}
\usepackage[a4paper,left=2cm,right=2cm,top=2.5cm,bottom=2cm]{geometry}
\usepackage{amsfonts}
\usepackage{amsmath}
\usepackage{amssymb}
\usepackage{amsthm}
\usepackage{babel}
\usepackage{hyperref}
\usepackage{cleveref}
\usepackage{color}
\usepackage{dsfont}
\usepackage{enumitem}
\usepackage{graphicx}
\usepackage{natbib}
\usepackage{pifont}

\theoremstyle{plain}% default
\newtheorem{thm}{Théorème}[section]
\newtheorem{lem}[thm]{Lemme}
\newtheorem{prop}[thm]{Proposition}
\newtheorem*{cor}{Corollaire}
\theoremstyle{definition}
\newtheorem{dfnt}{Définition}[section]
\newtheorem{exmp}{Exemple}[section]
\newtheorem{xca}[exmp]{Exercice}
\theoremstyle{remark}
\newtheorem*{rmq}{Remarque}
\newtheorem*{note}{Note}
\newtheorem{case}{Case}

\crefname{thm}{théorème}{théorèmes}
\crefname{lem}{lemme}{lemmes}
\crefname{prop}{proposition}{propositions}
\crefname{cor}{corollaire}{corollaires}
\crefname{dfnt}{définition}{définitions}
\crefname{exmp}{exemple}{exemples}
\crefname{xca}{exercice}{exercices}
\crefname{rmq}{remarque}{remarques}
\crefname{note}{note}{notes}
\crefname{case}{case}{cases}

\newcommand{\ncref}[1]{\cref{#1} "\nameref{#1}"}
\newcommand{\Ncref}[1]{\Cref{#1} \Nameref{#1}}

\begin{document}
Algorithme de Perron-Frobeinus

\section{Introduction: cas de $\mathbb{R}^{2}$}

Soit $x \in ]0;1[$ et $e_1$ $e_2$ une base de $\mathbb{R}^2$
On note $D$ la demi-droite passant par $(0;0)$ et $X=(x;1)$
Comment trouver un cône plus petit contenant $D$?

On note $X=xe_1+e_2=x(e_1+\frac{1}{x}e_2)=x(e_1+[\frac{1}{x}]e_2+{\frac{1}{x}}e_2)$
On pose donc $e_3=e_1+[\frac{1}{x}]e_2$ et nous avons $X=x(Txe_2+e_3)$ avec $Tx={\frac{1}{x}}$

Nous avons donc une suite de base $(e_n;e_{n+1})$ avec pour matrice de changement de base $P_n=
\begin{pmatrix}
   0 & 1 \\
   2 & [\frac{1}{T^{n-1}x}]
\end{pmatrix}$
En écrivant $e_{n+2}=(p_n;q_n)\in \mathbb{Z}^2$ nous avons:
$P_1...P_n=
\begin{pmatrix}
  p_{n-1} & p_n \\
  q_{n-1} & q_n
\end{pmatrix}$

Finalement nous avons $x=\frac{p_{n-1}T^n {x}+p_n}{q_{n-1}T^n x+q_{n-1}}$
\section{Quelques définitons}

$\tilde{w_i^n}=(\frac{c_{i,n}^n}{c^n_{i,d+1}},...,\frac{c^n_{i,d}}{c^n_{i,d+1}})$
Prenons $\theta \in \mathbb{R}^d$ et une approximation $\tilde{w}=(\frac{p_1}{q},...,\frac{p_d}{q})$
\begin{dfnt}
  Nous définissons l'exposent de Roth $\eta (w)=\eta(w,\theta)$ qui est:
  $\eta(w,\theta)=-\frac{log\| \theta - \tilde{w} \|}{log(q(w))}$
\end{dfnt}
\begin{dfnt}
  L'exponent de la meilleur approximitation est:
  $\eta(\theta)=limsup_{q(w) \to \infty} \eta(w,\theta)$
\end{dfnt}
Nous avons de même l'approximation uniforme qui concerne des matrices $W \in Gl(d+1,\mathbb{Z}), W=\begin{pmatrix}
w_1\\
. \\
. \\
. \\
w_{d+1}
\end{pmatrix}$

\begin{dfnt}
  L'exposent uniforme de $W$ pour $\theta$ est:
  $\eta^*(W,\theta)=\underset{1 \leq i \leq d+1}{min}(-\frac{log \| \theta - \tilde{w_i}}{log q(W)} \|)$
\end{dfnt}

\begin{dfnt}
  La meilleur approximation de matrice avec un denominateur inferieur à $Q$ est:
  $\eta^{*}(\theta,Q)=\underset{q(W) \leq Q}{inf} (\eta^{*}(W,\theta)\frac{log q(W)}{log Q})$
\end{dfnt}

\begin{dfnt}
  L'exposent de l'approximation uniforme pour $\theta$ est:
  $\eta^*(\theta)=\underset{Q \to \infty}{lim inf} \eta^*(\theta,Q)$
\end{dfnt}

De la même façon pour un algorithme MCF $A$ on peut définir,

\begin{dfnt}
  Le meilleur exposant pour $\theta$ de l'approximation en utilisant $A$:
  $\eta_A(\theta)=\underset{n \to \infty}{limsup}(\underset{1 \leq i \leq d+1}{max} \eta(w^n_i(\theta)))$
\end{dfnt}

\begin{dfnt}
  L'exposant uniforme pour $\theta$ de l'approximation en utilisant $A$:
  $\eta^*_A(\theta)=\underset{n \to \infty}{liminf}(\underset{1 \leq i \leq d+1}{min} \eta(w^n_i(\theta)))$
\end{dfnt}

\section{Convergence des algorithmes}

Un algorithme est dis faiblement convergent si $w_i^(n)\to \theta = (\theta_1,...,\theta_d)$ pour $1 \leq i \leq d+1$
Géométriquement l'angle entre les vecteurs et la demi-droite considérés tend vers $0$.

Un algorithme est dis fortement convergent si $|c_{i,d+1}^{n}(\theta-w_i^(n)) \to 0$
Les vecteurs deviennent arbitrairement proches de la demi-droite

\section{Oseledec's Multiplicative Ergodic Theorem}

\begin{thm}
Soit $(X,\zeta,d\mu)$ un espace de probabilité et $T : X \to X$ une application $\zeta$-mesurable et préservant la mesure.
Soit $A:X \to GL(d,\mathbb{R})$ mesurable, et posons $A^(n)(x)=A(T^(n-1)(x))...A(T(x))A(x)$
Supposons que $\mathbb{E}(log(max(\|A\|,1)))<\infty$
Alors il existe un ensemble mesurable $Z$ de mesure $1$ tel que $\forall x \in Z$ il y a
\begin{enumerate}
      \item un entier $r(x)$ tel que $0<r(x)<d$
      \item $r(x)$ différents nombre $\lambda_{r(x)}(x)<...<\lambda_1(x)$
      \item un ensemble de sous-espaces vectoriels $\mathbb{R}^d = E_1(x) \supsetneqq E_2(x)... \supsetneqq E_{r(x)} \supsetneqq E_{r(x)+1}={0}$ tel que si $v \in E_i(x)-E_{i+1}(x) \leftrightarrow lim_{n \to \infty} \frac{1}{n} log \|A^n (x)v\|= \lambda_i (x)$
      \item la limite $lim_{n \to infty} [A^n(x) {}^t A^n(x)]^{\frac{1}{n}}=A^{\infty}(x)$ existe et ses valeurs propres sont $e^{\lambda_i(x)}$ avec multiplicité $d_i$ avec comme espace propre $E_i(x)$
\end{enumerate}
Si en plus $T$ est ergodique pour $d \mu$ alors il y a un ensemble $Z'$ de mesure 1 tel que $r(x),\lambda_1(x),...,\lambda_{r(x)}(x),dim E_1(x),..., dim E_{r(x)}(x)$ sont constants $r,\lambda_1,\lambda_r,d_1,...,d_r$
\end{thm}

\section{Convergence de l'algorithme de Brun}

\begin{dfnt}
Soit $B={(x_1,x_2); 1 \geq x_1 \geq x_2 \geq 0}$, l'algorithme de Brun est généré par la fonction $S:B(j,N) \to B$ avec
$$
\left \{
\begin{array}{r c }
S(x_1,x_2)=(\frac{1}{x_1}-N,\frac{x_2}{x_1}) \, si \, \frac{1}{x_1}-N \geq \frac{x_2}{x_1} & [j=1] \\
S(x_1,x_2)=(\frac{x_2}{x_1},\frac{1}{x_1}-N) \, si \, \frac{x_2}{x_1} \geq  \frac{1}{x_1}-N & [j=2] \\
N:=[\frac{1}{x_1}]
\end{array}
\right .
$$
\end{dfnt}

\section{Unicité de la mesure $\tilde{\pi}$}
\begin{thm}
La mesure $\tilde{\pi}$ est l'unique mesure de probalité sur $\Delta$ qui se prette sur la mesure $\pi$ sue $X$ et qui est invariante par $\tilde{P}$
\end{thm}

\begin{lem}
Soit $\omega_n=(a_1,...,a_n)$ admissible. Alors la loi $\eta_{\omega_n}$ de $x_n$ sachant $\omega_n$ a pour densité (par rapport à $\pi$):$$
\frac{d \eta_{\omega_n}}{d \pi}(x)=\frac{p(x,\omega_n)}{\int_X p(x,\omega_n) d \pi (x)}
$$
\end{lem}
\begin{dfnt}
On appelle $S_(f)$ l'ensemble des suites admissibles de longueur finie. On note $M$ l'adhérence de l'ensemble des mesures $\eta_\omega$ quand $\omega$ parcourt $S_(f)$
\end{dfnt}

\begin{lem}
Si $\eta$ est dans $M$; alors $\eta$ admet une densité par rapport à $\pi$ qui est une fonction analytique sur $X$. $M$ est relativement compact pour la norme des variations des mesures.
\end{lem}

\begin{lem}
Soit $a$ dans $A_2$; pour $\pi$-presque tout $x$ de $X$, si $p(x,a)$ n'est pas nul, on a:$$
p(Tx,a)\frac{d a* \pi}{d \pi}(x)=
\left \{
\begin{array}{r c }
0 \, si \,  a \ne a^x \\
1 \, si \, a=a^x
\end{array}
\right .
$$
\end{lem}

\begin{lem}
Soit $\tilde{\nu}$ une mesure sur $\delta$ de la forme $\tilde{\nu}(f)=\int_\delta f(x,u)\nu_x(du)\pi(dx)$ Alors $\tilde{\nu}$ est invariante par $\tilde(P)$ si et seulement si pour $\pi$-presque tout $x$ de $X$, on a:$$
a^x*\nu_{Tx}=\nu_x
$$
\end{lem}

\begin{lem}
Soit $p$ un point de $P^2$, soit $\tilde{\nu}$ une mesure de probabilité sur $\delta$ qui est invariante par $\tilde(P)$. Soit $\sigma$ une permutation de $\Sigma_2$ et $W_\sigma$ la partie de $X$ correspondante. Alors sur un ensemble non négligeable de $W_\sigma$ on a: $\nu_x(\tilde{p})<1 $
\end{lem}

Définissons maintenant $\nu'_{\omega_n}(f)=\int_X \nu_x f d\nu_{\eta_{\omega_n}(x)}=\frac{\int_X f(x,u) \nu_x (du) p(x,\omega_n) d \pi(x)}{\int_X p(x,\omega_n) d \pi(x)}$
Nous avons alors la proposition:

\begin{prop}
Soit $x \in X$ dont le dévellopemment de Jacobi-Perron est $(a_n)$, soit $\nu$ une mesure de probabilité sur $\Delta$ in variante par l'action de $\tilde{P}$, alors $a_1 ... a_n \nu'_{\omega_n}=\omega_n \nu'_{\omega_n}$ est une martingale qui converge $\pi$-presque partout vers $\nu_x$.
De plus pour $\pi$- presque tout $x,\zeta_x$ est un atome de $\nu_x$
\end{prop}

\section{Propieté de contraction de $\tilde{P}$}

\begin{lem}
La série $\sum_{\omega_n} [log(\omega_n,\zeta)]^2 e^{\epsilon |log(\omega_n,\zeta)|}p(x,\omega_n)$ converge uniformément sur $\bar{M}$
\end{lem}

\begin{lem}
$\exists \gamma \in \mathbb{R}$, avec $\lambda_1 \leq \gamma < 0$ tel que pour $n$ suffisament grand nous ayons:$$
\sum_{\omega_n} log(\theta(\omega_n,\zeta_n))p(x,\omega_n,\sigma_n) \leq n \gamma <0
$$
\end{lem}

\section{Equations de la preuve de Schratzberger}
(1)
$\Omega_B^{(t)}=\begin{pmatrix}
q^{(t)} & q^{(t')} & q^{(t'')} \\
p_1^{(t)} & p_1^{(t')} & p_1^{(t'')}\\
p_2^{(t)} & p_2^{(t')} & p_2^{(t'')}
\end{pmatrix}
:= \Omega_B^{(t-1)} \Lambda_B^{(t-1)}$
\newline
(2)
$x_i^(0)=\frac{p_i^{(t)}+x_1^{(t)}p_i^{(t')}+x_2^{(t)}p_i^{(t'')}}{q^{(t)}+x_1^{(t)}q^{(t')}+x_2^{(t)}q^{(t'')}}$
\newline
(3)
if $j(t)=1$ $q^{(t+2)}=N^{(t+1)}q^{(t+1)}+q^{(t)}$
\newline
(4)
if $j(t)=2$ $q^{(t+2)}=N^{(t+1)}q^{(t+1)}+q^{(t^*)}$
\newline
(5)
(6)
(7)
(8)
(9)
(10)
(11)
(12)\cite{ref}
\bibliographystyle{plain}
\bibliography{biblio.bib}
\end{document}
